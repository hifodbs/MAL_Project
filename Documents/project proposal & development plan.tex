\documentclass[a4paper]{article}

% -------------------------------------------------
% Packages
% -------------------------------------------------
\usepackage[T1]{fontenc}
\usepackage{lmodern}
\usepackage{geometry}
\usepackage{setspace}
\usepackage[hidelinks]{hyperref}
\usepackage{graphicx}
\usepackage{longtable}
\usepackage{booktabs}
\usepackage{listings}
\usepackage{xcolor}
\usepackage{enumitem}
\usepackage{fancyhdr}

% -------------------------------------------------
% Page Layout
% -------------------------------------------------
\geometry{left=25mm,right=25mm,top=25mm,bottom=25mm}
\setstretch{1.15}

% -------------------------------------------------
% Header / Footer
% -------------------------------------------------
\pagestyle{fancy}
\fancyhf{}
\fancyhead[L]{Project Proposal \& Development Plan}
\fancyhead[R]{\thepage}

% -------------------------------------------------
% Metadata
% -------------------------------------------------
\title{\textbf{System Specification Document}}
\author{S.O.L.A.R. Project}
\date{\today}

% -------------------------------------------------
% Document
% -------------------------------------------------
\begin{document}

\maketitle
\thispagestyle{empty}
\newpage

% -------------------------------------------------
% Revision History
% -------------------------------------------------
\section*{Revision History}
\begin{longtable}{@{}lll p{7cm}@{}}
\toprule
Version & Date & Author & Description \\ \midrule
0.1 & 2025-12-17 & Matteo & Initial draft \\
1.0 & 2026-01-25 & Matteo & Approved release \\
\bottomrule
\end{longtable}

\newpage
\tableofcontents
\newpage

% =================================================
% 1. Project Summary
% =================================================

\section{Project Summary}

\subsection{Scope}

The scope of this project is the design and development of an end-to-end data-driven 
system for short-term solar power generation forecasting. The system covers the full 
lifecycle of a machine learning application, including data ingestion, preprocessing, 
model training, validation, deployment, monitoring, and incremental updates. The 
solution is designed to simulate real-world operational conditions through synthetic sensors.
The project further incorporates a slightly simpler ILSTM model proposed in this 
\href{https://www.sciencedirect.com/science/article/pii/S0045790621001592}{paper} 
as a proof of concept for advanced forecasting techniques.

\subsection{Objectives}

The objectives of the project are the following:
\begin{itemize}
    \item Design and implement a reliable regression-based system for short-term solar power forecasting.
    \item Develop an incremental learning pipeline capable of adapting to evolving data distributions.
    \item Implement monitoring mechanisms for data drift and performance degradation.
    \item Integrate governance features such as metric logging, model evaluation, and reproducibility support.
    \item Provide a proof of concept for extending the system with an Incremental Long Short-Term Memory (ILSTM) model.
\end{itemize}

\subsection{Relevance}

This project is relevant in the context of renewable energy management, 
where accurate short-term forecasting plays a critical role in ensuring grid 
stability, optimizing energy dispatch, and reducing imbalance penalties. 
Reliable predictions of solar power generation enable energy operators to 
better align supply with demand and to improve the overall efficiency of power 
system operations.

The project demonstrates the practical application of machine learning and 
neural network techniques for time-series forecasting, focusing on the prediction of 
next-day solar power generation using historical production data and 
weather-related features. By addressing real-world challenges such as 
data variability and evolving system behavior, the project highlights the 
value of data driven solutions in supporting sustainable energy management.

% =================================================
% 2. Deliverables
% =================================================

\section{Deliverables}

\begin{itemize}
    \item A data ingestion and preprocessing pipeline supporting historical data.
    \item An incremental learning regression model for next-timestamp solar power forecasting.
    \item A server-side prediction service implemented in Flask, exposing a REST-style API.
    \item An interactive Streamlit dashboard for visualization of predictions and monitoring metrics.
    \item A proof-of-concept implementation and evaluation of an incremental LSTM (ILSTM) model.
    \item A DAO interface for querying sensor data, including support for synthetic sensor sources.
    \item Technical documentation describing system architecture, model design, and monitoring strategy.
\end{itemize}

% =================================================
% 3. Milestones
% =================================================
\subsection{Milestones}


\begin{enumerate}
    \item Completion of dataset analysis and validation of data quality.
    \item Development and evaluation of a baseline incremental regression model.
    \item Implementation of the DAO layer and integration of synthetic sensor data sources.
    \item Deployment of the Flask-based prediction service exposing a REST-style API.
    \item Development of the Streamlit dashboard for real-time visualization and monitoring.
    \item Activation of monitoring components for data drift and prediction performance.
    \item Implementation and evaluation of an advanced incremental LSTM (ILSTM) model.
\end{enumerate}




% =================================================
% 4. Work Breakdown Structure (WBS)
% =================================================
\subsection{Work Breakdown Structure}

The project started by a brainstorming session to identify the main components 
and tasks required to achieve the project objectives. The WBS was then structured 
into the following main components: 

% ad helping team member

\subsubsection{Project Manager}
\begin{itemize}
    \item Coordination of development activities and task scheduling.
    \item Redacting and maintaining project documentation.
    \item Review and validation of technical documentation and deliverables.
\end{itemize}

\subsubsection{Data Analyst / ML Engineer}
\begin{itemize}
    \item Exploratory data analysis and validation of the datasets.
    \item Development and evaluation of the baseline incremental regression model.
    \item Integration of monitoring components for drift detection and performance tracking.
    \item Design, implementation, and evaluation of the incremental LSTM (ILSTM) model.
    \item Definition and monitoring of performance metrics and KPIs.
\end{itemize}

\subsubsection{Software Developer}
\begin{itemize}
    \item Implementation of the DAO layer for accessing synthetic sensor data.
    \item Development of the Flask-based prediction service and REST-style API.
    \item Development of the Streamlit dashboard for visualization and monitoring.
\end{itemize}

% =================================================
% 5. Schedule
% =================================================
\subsection{Sprint Plan}

Due to the vacation period, the length of the sprints has been adjusted to accommodate
the team members' availability. The sprint plan is as follows:

\begin{center}
\begin{tabular}{p{1.5cm} p{2cm} p{11cm}}    
\toprule
Sprint & Start Date & Activities \\
\midrule
Sprint 0 & -- &
Brainstorming and definition of project requirements, along with analysis of the historical dataset. \\

Sprint 1 & 19-12-2025 &
Implementation of the core system backbone, including the Streamlit dashboard, Flask service, and DAO layer; preprocessing of historical data and training of the first incremental regression model. \\

Sprint 2 & 23-12-2025 &
API refinement, integration of the synthetic sensor simulation, and delivery of the first end-to-end basic model. \\

Sprint 3 & 16-01-2026 &
Development of the Minimum Viable Product (MVP), integration of ADWIN-based drift detection, and implementation of the incremental LSTM (ILSTM) model. \\

Sprint 4 & 23-01-2026 &
Final model tuning, full system deployment, and completion of monitoring and documentation. \\
\bottomrule
\end{tabular}
\end{center}

% =================================================
% 6. Definition of Done (DoD) \& Definition of Ready (DoR)
% =================================================
\subsection{Definition of Ready (DoR)}

A task or feature is considered ready to start when all the following conditions are met:

\begin{itemize}
    \item The task is clearly defined, with measurable objectives and scope.
    \item Required datasets are available, cleaned, and preprocessed for use with synthetic sensors.
    \item Technical specifications, including API endpoints, dashboard requirements, and model input/output, are documented.
    \item Dependencies, such as necessary libraries (Flask, Streamlit, River), frameworks, or environment setup, are identified and accessible.
    \item Acceptance criteria are clearly defined, including performance targets for the incremental regression model and drift detection.
\end{itemize}

\subsection{Definition of Done (DoD)}

A task or feature is considered done when all the following conditions are satisfied:

\begin{itemize}
    \item The model is implemented, integrated into the Flask API, and correctly returns predictions to the dashboard.
    \item The synthetic sensor interface is functional, providing real-time simulated data for predictions.
    \item Incremental learning updates the model continuously, and ADWIN drift detection monitors feature distributions and triggers alerts appropriately.
    \item Model performance meets the predefined KPIs, such as daily mean MSE below the target threshold.
    \item The codebase is fully documented, versioned, and reviewed according to project standards.
    \item Basic functional tests of the entire pipeline (API, dashboard, monitoring, and sensor interface) have passed successfully.
    \item All related documentation, including design decisions and system architecture, is completed and available.
\end{itemize}

% =================================================
% 7. Resources \& Infrastructure
% =================================================
\subsection{Resources and Infrastructure}

\begin{itemize}
    \item \textbf{Programming Languages and Libraries:}
    \begin{itemize}
        \item Python: Core language for data processing, modeling, and backend development.
        \item Libraries: pandas, numpy, scikit-learn, river (for real-time machine learning), 
        Flask (for REST API), Streamlit (for dashboard visualization), TensorFlow and Keras (for ILSTM model implementation).
    \end{itemize}

    \item \textbf{Data Management:}
    \begin{itemize}
        \item Raw datasets from solar power plants, including sensor data for temperature, irradiation, and power generation are stored as immutable objects. 
        \item Clean CSV files for model training are generating through deterministic preprocessing.
        \item At runtime DAOs retrive data for the server. 
    \end{itemize}

    \item \textbf{Version Control:}
    \begin{itemize}
        \item GitHub: Used for version control, collaboration, and documentation.
    \end{itemize}

    \item \textbf{Development Tools:}
    \begin{itemize}
        \item Jupyter Notebooks: For exploratory data analysis and prototyping.
        \item VS Code: Integrated development environment for coding and debugging.
    \end{itemize}

    \item \textbf{Infrastructure:}
    \begin{itemize}
        \item Local development environment for testing and debugging.
    \end{itemize}

    \item \textbf{Collaboration Tools:}
    \begin{itemize}
        \item Documentation: LaTeX for project documentation and reporting.
    \end{itemize}
\end{itemize}

\end{document}
