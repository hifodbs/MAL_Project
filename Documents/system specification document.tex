\documentclass[a4paper]{article}

% -------------------------------------------------
% Packages
% -------------------------------------------------
\usepackage[T1]{fontenc}
\usepackage{lmodern}
\usepackage{geometry}
\usepackage{setspace}
\usepackage{hyperref}
\usepackage{graphicx}
\usepackage{longtable}
\usepackage{booktabs}
\usepackage{array}
\usepackage{listings}
\usepackage{xcolor}
\usepackage{enumitem}
\usepackage{fancyhdr}

% -------------------------------------------------
% Page Layout
% -------------------------------------------------
\geometry{
  left=25mm,
  right=25mm,
  top=25mm,
  bottom=25mm
}

\setstretch{1.15}

% -------------------------------------------------
% Header / Footer
% -------------------------------------------------
\pagestyle{fancy}
\fancyhf{}
\fancyhead[L]{System Specification Document}
\fancyhead[R]{\thepage}

% -------------------------------------------------
% Code Listings
% -------------------------------------------------
\lstset{
  basicstyle=\ttfamily\small,
  frame=single,
  breaklines=true,
  backgroundcolor=\color{gray!5}
}

% -------------------------------------------------
% Metadata
% -------------------------------------------------
\title{\textbf{System Specification Document}}
\author{S.O.L.A.R. Project}
\date{\today}

% -------------------------------------------------
% Document
% -------------------------------------------------
\begin{document}

\maketitle
\thispagestyle{empty}
\newpage

% -------------------------------------------------
% Revision History
% -------------------------------------------------
\section*{Revision History}
\begin{longtable}{@{}lll p{7cm}@{}}
\toprule
Version & Date & Author & Description \\ \midrule
0.1 & 2025-12-17 & Matteo & Initial draft \\
1.0 & YYYY-MM-DD & Name & Approved release \\
\bottomrule
\end{longtable}

\newpage
\tableofcontents
\newpage

% =================================================
% 1. Introduction
% =================================================
\section{Problem Definition}

\subsection{Business Problem}

Solar power plants are subject to strong variability due to weather 
conditions. 
Inaccurate short-term forecasts of power generation may cause grid 
imbalance penalties and inefficient energy dispatch.
The objective of this project is to provide accurate and reliable 
predictions of solar power output to support operational 
decision-making.


\subsection{ML Problem Formulation}

Given historical weather data (temperature, irradiance, humidity,
wind speed, cloud coverage),
predict the continuous value of electrical power output (kW) for 
a solar plant at a given time.

This is formulated as a supervised regression problem.

\subsection{Key Performance Indicators (KPIs)}

\begin{itemize}
    \item Root Mean Squared Error (RMSE)
    \item Mean Absolute Error (MAE)
    \item Coefficient of Determination ($R^2$)
    \item Business KPI: reduction of forecasting error compared to a 
    persistence baseline
\end{itemize}

% =================================================
% 2. Data Specification
% =================================================
\section{Data Specification}

\subsection{Data Source}

\begin{itemize}
    \item Weather variables: temperature, solar irradiance, 
    humidity, wind speed
    \item Target variable: generated power (kW)
    \item Timestamp information
\end{itemize}


\subsubsection{Data Flow}
\begin{enumerate}
    \item Raw data ingestion from CSV files
    \item Schema validation and missing value checks
    \item Feature engineering and normalization
    \item Dataset splitting into training, validation, and test sets
\end{enumerate}

\subsubsection{Data Quality and Preprocessing}
Missing values are handled through interpolation or forward-filling.
Outliers caused by sensor malfunctions are detected using statistical 
thresholds.
Numerical features are standardized to improve model convergence.
Temporal features such as hour of day and day of year are extracted.


% =================================================
% 3. Functional Requirements
% =================================================
\section{Functional Requirements}

\subsection{Use Cases}
See in real time the power produced by the power plant.
See how much power will be predicted in the same day
and next day

\subsection{Functional Requirement List}
\begin{longtable}{@{}p{2cm} p{11cm}@{}}
\toprule
ID & Description \\ \midrule
FR-01 & The system shall ingest historical data.\\
FR-02 & The model shall forecast timestamp predictions for the next 24 hours.\\
FR-03 & The system shall forecast total energy yield for the next 24 hours.\\
FR-04 & The system shall detect anomalies in the power production.\\
FR-05 & The system shall identify underperforming inverters.\\
FR-06 & The system shall display the list of underperforming inverters and their efficiency loss.\\
FR-07 & The systems shall display the plot for the total predicted power generation against the actual one.\\
FR-08 & The systems shall display the plot for the predicted power generation for each timestamp against the actual one.\\

\bottomrule
\end{longtable}

% =================================================
% 4. Non-Functional Requirements
% =================================================
\section{Non-Functional Requirements}

\begin{longtable}{@{}p{2cm} p{11cm}@{}}
\toprule
ID & Description \\ \midrule
NFR-01 & The forecasting model shall achieve a Mean Absolute Error 
(MAE) of less than a threshold on the validation set.\\
NFR-02 & The anomaly detection algorithm shall have a low False 
Positive Rate (FPR) to prevent unnecessary maintenance dispatch.\\
\bottomrule
\end{longtable}


% =================================================
% 5. System Architecture
% =================================================

\subsection{System Architecture}

\subsubsection{Training}
Offline model training is performed using scikit-learn pipelines.

\subsubsection{Validation}
Cross-validation is applied to reduce overfitting and assess 
generalization.

\subsubsection{Deployment}
The trained model is containerized using Docker and deployed via a 
FastAPI service.

\subsubsection{Monitoring}
The system monitors prediction errors and detects feature drift using 
statistical tests.

% =================================================
% 6. Risk Analysis
% =================================================
\section{Risk Analysis}

\begin{center}
\begin{tabular}{lll}
\toprule
Risk & Impact & Mitigation \\
\midrule
Data drift & Model degradation & Periodic retraining \\
Missing data & Incorrect predictions & Validation checks \\
Overfitting & Poor generalization & Cross-validation \\
\bottomrule
\end{tabular}
\end{center}

\end{document}
