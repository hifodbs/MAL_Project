\documentclass[a4paper]{article}

% -------------------------------------------------
% Packages
% -------------------------------------------------
\usepackage[T1]{fontenc}
\usepackage{lmodern}
\usepackage{geometry}
\usepackage{setspace}
\usepackage{hyperref}
\usepackage{graphicx}
\usepackage{longtable}
\usepackage{booktabs}
\usepackage{array}
\usepackage{listings}
\usepackage{xcolor}
\usepackage{enumitem}
\usepackage{fancyhdr}

% -------------------------------------------------
% Page Layout
% -------------------------------------------------
\geometry{
  left=25mm,
  right=25mm,
  top=25mm,
  bottom=25mm
}

\setstretch{1.15}

% -------------------------------------------------
% Header / Footer
% -------------------------------------------------
\pagestyle{fancy}
\fancyhf{}
\fancyhead[L]{System Specification Document}
\fancyhead[R]{\thepage}

% -------------------------------------------------
% Code Listings
% -------------------------------------------------
\lstset{
  basicstyle=\ttfamily\small,
  frame=single,
  breaklines=true,
  backgroundcolor=\color{gray!5}
}

% -------------------------------------------------
% Metadata
% -------------------------------------------------
\title{\textbf{System Specification Document}}
\author{S.O.L.A.R. Project}
\date{\today}

% -------------------------------------------------
% Document
% -------------------------------------------------
\begin{document}

\maketitle
\thispagestyle{empty}
\newpage

% -------------------------------------------------
% Revision History
% -------------------------------------------------
\section*{Revision History}
\begin{longtable}{@{}lll p{7cm}@{}}
\toprule
Version & Date & Author & Description \\ \midrule
0.1 & 2025-12-17 & Matteo & Initial draft \\
1.0 & YYYY-MM-DD & Name & Approved release \\
\bottomrule
\end{longtable}

\newpage
\tableofcontents
\newpage

% =================================================
% 1. Introduction
% =================================================
\section{Introduction}

\subsection{Purpose}
The purpose of the project is to optimize the maintenance and output of solar power plants. Utilizing historical 
power generation and weather sensor data, the system 
will provide two critical capabilities: predicting energy production for the upcoming 24 hours and identifying 
underperforming inverters caused by dirty panels or faulty equipment. The system targets Plant Managers and 
Maintenance Engineers to facilitate proactive maintenance and grid load planning.
.

\subsection{Scope}
The system should give a website from what is possible to see power 
production in real time and how much was predicted for the same day
and the day after.


\subsection{Definitions, Acronyms, and Abbreviations}
\begin{description}
  \item[API] Application Programming Interface
\end{description}

\subsection{References}
List applicable standards, documents, or specifications.
NO IDEA

\subsection{ML formulation}
ML expert need to fill this section.

\subsection{KPIs}
The performance to be monitored are the power predicted the same day
with the actual power produced.


% =================================================
% 2. Data Specification
% =================================================
\section{Data Specification}
The data is gather from a dataset found on Kaggle. For simulating
a real system there will be two machine that will generate the unseen
data. One will generate the data by the solar pannels and the other
will generate the weather condition. The dataset is complete and
doesn't contain errors. But we seen a mismatch of dimension in one
column so it needed to be adapted to the context.

% =================================================
% 2. System Overview
% =================================================
\section{System Overview}

\subsection{System Description}
The system follows a data pipeline architecture comprising Data Ingestion, Machine Learning (ML) Processing, and a 
Visualization Dashboard.
The main modules include:
Data Ingestion Module: Handles the loading and cleaning of generation and weather sensor data (based on the provided 
Kaggle dataset schema).
Forecasting Module: Utilizes regression models to predict AC/DC power output based on weather parameters (irradiation, 
temperature).
Anomaly Detection Module: Identifies deviations between expected and actual power output to flag specific source keys 
(inverters) requiring maintenance.
Dashboard Module: A web-based interface for visualizing forecasts and active alerts.


% =================================================
% 3. Functional Requirements
% =================================================
\section{Functional Requirements}

\subsection{Use Cases}
See in real time the power produced by the power plant.
See how much power will be predicted in the same day
and next day

\subsection{Functional Requirement List}
\begin{longtable}{@{}p{2cm} p{11cm}@{}}
\toprule
ID & Description \\ \midrule
FR-01 & The system shall ingest historical data.\\
FR-02 & The model shall forecast timestamp predictions for the next 24 hours.\\
FR-03 & The system shall forecast total energy yield for the next 24 hours.\\
FR-04 & The system shall detect anomalies in the power production.\\
FR-05 & The system shall identify underperforming inverters.\\
FR-06 & The system shall display the list of underperforming inverters and their efficiency loss.\\
FR-07 & The systems shall display the plot for the total predicted power generation against the actual one.\\
FR-08 & The systems shall display the plot for the predicted power generation for each timestamp against the actual one.\\

\bottomrule
\end{longtable}

% =================================================
% 4. Non-Functional Requirements
% =================================================
\section{Non-Functional Requirements}

\begin{longtable}{@{}p{2cm} p{11cm}@{}}
\toprule
ID & Description \\ \midrule
NFR-01 & The forecasting model shall achieve a Mean Absolute Error (MAE) of less than a threshold on the validation set.\\
NFR-02 & The anomaly detection algorithm shall have a low False Positive Rate (FPR) to prevent unnecessary maintenance dispatch.\\
\bottomrule
\end{longtable}


% =================================================
% 5. System Architecture
% =================================================
\section{System Architecture}

\subsection{Architectural Overview}
Describe architecture style (e.g., client-server, microservices).

\subsection{Component Diagram}
\begin{figure}[h]
  \centering
  %\includegraphics[width=0.8\textwidth]{architecture.png}
  \caption{System Architecture Diagram}
\end{figure}

\subsection{Component Descriptions}
Describe each major component and responsibility.

% =================================================
% 6. Interfaces
% =================================================
\section{Interfaces}

\subsection{External Interfaces}
APIs, third-party services, protocols.

\subsection{Internal Interfaces}
Inter-module communication.

\subsection{API Specification Example}
\begin{lstlisting}
POST /device/register
{
  "otp": "string",
  "deviceFingerprint": "string"
}
\end{lstlisting}

% =================================================
% 7. Data Design
% =================================================
\section{Data Design}

\subsection{Data Model}
High-level entity relationships.

\subsection{Database Schema}
\begin{longtable}{@{}lll p{6cm}@{}}

\toprule
Field & Type & Required & Description \\ \midrule
%device_id & UUID & Yes & Unique device identifier \\
%otp_hash & String & Yes & Hashed OTP value \\
\bottomrule
\end{longtable}

% =================================================
% 8. Error Handling and Logging
% =================================================
\section{Error Handling and Logging}

\subsection{Error Codes}
Describe error responses and codes.

\subsection{Logging Strategy}
Log levels, retention, and audit requirements.

% =================================================
% 9. Security Considerations
% =================================================
\section{Security Considerations}

Threat model, attack vectors, mitigations, and compliance.

% =================================================
% 10. Deployment and Environment
% =================================================
\section{Deployment and Environment}

\subsection{Deployment Architecture}
Environments (dev, staging, production).

\subsection{Configuration}
Environment variables, secrets management.

% =================================================
% 11. Appendices
% =================================================
\section{Appendix}

\subsection{Glossary}
Optional glossary.

\subsection{Future Enhancements}
Planned or potential improvements.

\end{document}
